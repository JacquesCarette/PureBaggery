\documentclass{article} % change to whatever later
\usepackage{fullpage}
\usepackage{amsmath,amsthm}

\theoremstyle{plain}
\newtheorem{theorem}{Theorem}

\theoremstyle{definition}
\newtheorem{example}[theorem]{Example}
\newtheorem{remark}[theorem]{Remark}

\newcommand{\PL}{PL} % expand out later
\newcommand{\BSp}{\(\mathbf{B}\)-Species}
\newcommand{\LSp}{\(\mathbf{L}\)-Species}

\title{Structured Positions}
\author{}

\begin{document}
\maketitle
\begin{abstract}
The well-supported data-structures, aka inductive types, in functional
languages are insufficiently rich. We can do better.
\end{abstract}

The starting-point question is: how can one properly teach a \PL about
bags (aka multisets)?

Adding a single new type is rather boring though. Even though doing it
properly is difficult (most languages have rather poor abstraction facilities
so that giving a proper API that doesn't \emph{leak} is hard), it is still
to \textit{ad hoc}. The desire is to find a family of ``type constructors''
that would contain both inductive types and bags.

And even that is aiming far too low. The real problem is 
\textbf{how to teach computers about structures}. In other words, if a
human performs the act of defining a particular structure, a lot of
natural \emph{kit} should come along ``for free'' as a side-effect of
that declaration of intent. (Haskell's \texttt{Deriving} mechanism is
a reasonable analogy; of course getting an induction principle should also
be within scope).

\section{Better Structures}

In a way, we already know that this is \texit{in theory} feasible:
species, \BSp and \LSp. In practice, the theory is insufficiently
constructive nor does it provide answers for key aspects of treating
a structure so obtained as a \emph{data} structure. In particular, no
eliminator.

Another approach is to use the theory of containers. Which, unfortunately,
don't give eliminators either!
\end{document}
